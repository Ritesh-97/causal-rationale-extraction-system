\documentclass[11pt,a4paper]{article}
\usepackage[utf8]{inputenc}
\usepackage[T1]{fontenc}
\usepackage{amsmath,amsfonts,amssymb}
\usepackage{graphicx}
\usepackage{hyperref}
\usepackage{geometry}
\usepackage{listings}
\usepackage{xcolor}
\usepackage{booktabs}
\usepackage{multirow}
\usepackage{algorithm}
\usepackage{algpseudocode}

\geometry{margin=1in}

\title{Causal Rationale Extraction and Synthesis from Conversational Data on Business Events}
\author{Team Name}
\date{\today}

\begin{document}

\maketitle

\begin{abstract}
This report presents a comprehensive system for extracting causal rationales from large-scale conversational transcripts to identify relationships between dialogue dynamics and business events such as escalations, refunds, and churn. The system implements a two-task framework: (1) query-driven evidence-based causal explanation generation, and (2) contextual follow-up conversation support. Our approach combines retrieval-augmented generation (RAG), semantic search, reranking, and causal pattern detection to provide interpretable, evidence-based explanations. We evaluate the system using a curated dataset of 50-100 queries across multiple event types and demonstrate its effectiveness through quantitative metrics and qualitative analysis.
\end{abstract}

\section{Introduction}

\subsection{Problem Statement}
In large-scale customer-interaction operations, contact centers process tens of thousands to hundreds of thousands of agent-customer dialogues. Many conversations culminate in business-critical outcomes such as customer escalations, refund requests, or signals of churn. These events carry significant cost, risk, and operational overhead. However, current monitoring systems often flag that an adverse event occurred but provide little insight into why it happened.

The challenge lies in identifying which turns or segments of dialogue triggered the event, which conversational cues systematically lead to specific outcomes, and what temporal patterns presage certain events. Without this visibility, organizations cannot systematically perform root-cause analysis, coach agents precisely, redesign processes, or intervene proactively.

\subsection{Objectives}
This work aims to develop a system that:
\begin{itemize}
    \item Processes large volumes of transcript data with speaker labels and turn indexing
    \item Models conversational dynamics and maps dialogue flows to business events
    \item Surfaces specific dialogue spans that most likely causally contributed to events
    \item Enables analytic querying across call corpora to identify recurring causal motifs
    \item Provides evidence-based, interpretable explanations for business events
    \item Supports contextual follow-up conversations for iterative analysis
\end{itemize}

\section{Related Work}

Previous work in conversational analysis has focused on sentiment analysis, topic modeling, and event detection. However, causal rationale extraction from dialogue remains underexplored. Our approach builds upon:
\begin{itemize}
    \item \textbf{Retrieval-Augmented Generation (RAG)}: Combining dense retrieval with language models for grounded generation
    \item \textbf{Causal Inference in NLP}: Methods for identifying causal relationships in text
    \item \textbf{Dialogue Analysis}: Techniques for understanding conversational dynamics
    \item \textbf{Explainable AI}: Approaches for providing interpretable explanations
\end{itemize}

\section{System Architecture}

\subsection{Overview}
Our system consists of six main components:
\begin{enumerate}
    \item \textbf{Data Processing Pipeline}: Transcript ingestion, preprocessing, and vector database indexing
    \item \textbf{Retrieval System}: Semantic search, reranking, and dialogue span extraction
    \item \textbf{Causal Analysis Module}: Pattern detection, evidence scoring, and causal span identification
    \item \textbf{Query Processing}: Natural language understanding and intent classification
    \item \textbf{Explanation Generation}: LLM-based synthesis with evidence citation
    \item \textbf{Conversation Management}: Context tracking and follow-up handling
\end{enumerate}

\subsection{Data Processing Pipeline}

\subsubsection{Transcript Loading}
The system supports multiple transcript formats:
\begin{itemize}
    \item \textbf{JSON}: Structured format with turns, events, and metadata
    \item \textbf{CSV}: Tabular format with columns for transcript\_id, turn\_id, speaker, text, timestamp, event\_type, event\_label
    \item \textbf{TXT}: Simple text format with basic parsing
\end{itemize}

\subsubsection{Preprocessing}
The preprocessing module performs:
\begin{itemize}
    \item Speaker label normalization (agent/customer)
    \item Text cleaning and normalization
    \item Turn segmentation and indexing
    \item Event type normalization
    \item Dialogue structure extraction
\end{itemize}

\subsubsection{Vector Database Indexing}
Dialogue spans are extracted using sliding windows (default: 5 turns per span) and indexed into ChromaDB for efficient retrieval. Each span includes:
\begin{itemize}
    \item Combined text from multiple turns
    \item Turn indices and IDs
    \item Speaker distribution
    \item Event associations (if applicable)
\end{itemize}

\subsection{Retrieval System}

\subsubsection{Semantic Search}
We use sentence transformers (all-MiniLM-L6-v2) to encode queries and dialogue spans into dense vector representations. Cosine similarity is used to retrieve the top-k most relevant spans.

\subsubsection{Reranking}
A cross-encoder (ms-marco-MiniLM-L-6-v2) reranks the retrieved spans based on query relevance. This two-stage approach improves precision by refining semantic search results.

\subsubsection{Event-Specific Retrieval}
For queries about specific event types, the system filters spans that are associated with those events and extracts causal spans around event occurrences.

\subsection{Causal Analysis Module}

\subsubsection{Pattern Detection}
The pattern detector identifies:
\begin{itemize}
    \item \textbf{Temporal patterns}: Spans that precede events
    \item \textbf{Sequential patterns}: Consecutive spans with causal relationships
    \item \textbf{Behavioral patterns}: Indicators like hesitation, frustration, repetition
    \item \textbf{Event-specific patterns}: Triggers for escalations, refunds, churn
\end{itemize}

\subsubsection{Evidence Scoring}
Evidence spans are scored using a weighted combination of:
\begin{itemize}
    \item \textbf{Relevance score} (40\%): From reranking
    \item \textbf{Temporal score} (30\%): Proximity to event
    \item \textbf{Pattern score} (20\%): Causal pattern indicators
    \item \textbf{Similarity score} (10\%): Semantic similarity to query
\end{itemize}

\subsection{Explanation Generation}

\subsubsection{LLM Integration}
The system supports multiple LLM providers:
\begin{itemize}
    \item OpenAI (GPT-4, GPT-3.5)
    \item Anthropic (Claude)
    \item Google Gemini (gemini-pro)
\end{itemize}

\subsubsection{Prompt Engineering}
Prompts are structured to:
\begin{itemize}
    \item Provide context about the task
    \item Include formatted evidence spans
    \item Request structured explanations with citations
    \item Maintain conversational context for follow-ups
\end{itemize}

\subsection{Conversation Management}

\subsubsection{Context Tracking}
The conversation manager maintains:
\begin{itemize}
    \item Conversation history (recent turns)
    \item Query-response pairs
    \item Context summaries for follow-up queries
\end{itemize}

\subsubsection{Follow-up Detection}
Follow-up queries are identified by:
\begin{itemize}
    \item Presence of follow-up indicators (also, furthermore, what about)
    \item Pronoun usage (it, that, this, these)
    \item Query length (short queries often follow-ups)
    \item Contextual references to previous queries
\end{itemize}

\section{Methodology}

\subsection{Task 1: Query-Driven Evidence-Based Causal Explanation}

For initial queries, the system:
\begin{enumerate}
    \item Parses the query to extract event type and intent
    \item Retrieves relevant dialogue spans using semantic search
    \item Reranks spans by relevance
    \item Analyzes spans for causal patterns
    \item Scores evidence spans
    \item Generates explanation with LLM using top evidence
    \item Formats response with citations and metadata
\end{enumerate}

\subsection{Task 2: Conversational Follow-Up and Contextual Response}

For follow-up queries, the system:
\begin{enumerate}
    \item Detects if query is a follow-up
    \item Retrieves conversation context
    \item Enhances query with context information
    \item Performs context-aware retrieval
    \item Generates contextual explanation
    \item Updates conversation history
\end{enumerate}

\subsection{Query Dataset Generation}

We developed a query simulation framework that:
\begin{itemize}
    \item Uses LLMs to generate realistic queries from agent coach perspective
    \item Categorizes queries by task, difficulty, and use case
    \item Generates follow-up queries based on initial responses
    \item Applies human-in-the-loop refinement for quality
    \item Uses LLM-as-Judge for diversity and relevance assessment
\end{itemize}

\section{Evaluation}

\subsection{Evaluation Metrics}

We evaluate the system using multiple metrics:

\subsubsection{Response Quality}
\begin{itemize}
    \item Length and word count
    \item Sentence count
    \item Citation presence and count
    \item Coherence score (transition word usage)
\end{itemize}

\subsubsection{Evidence Quality}
\begin{itemize}
    \item Evidence count
    \item Average evidence score
    \item Evidence coverage (transcript diversity)
    \item Evidence diversity (span uniqueness)
\end{itemize}

\subsubsection{Causal Explanation Quality}
\begin{itemize}
    \item Causal language presence
    \item Causal indicator count
    \item Evidence reference count
    \item Explanation completeness (query keyword coverage)
\end{itemize}

\subsubsection{Conversational Coherence}
\begin{itemize}
    \item Coherence score
    \item Context usage
    \item Reference count to previous turns
    \item Context relevance
\end{itemize}

\subsection{Baseline Comparisons}

We compare against three baselines:
\begin{enumerate}
    \item \textbf{Keyword Search}: Simple keyword matching
    \item \textbf{Simple RAG}: Semantic search without reranking
    \item \textbf{Rule-Based}: Pattern matching with predefined rules
\end{enumerate}

\subsection{Ablation Studies}

We perform ablation studies by removing components:
\begin{itemize}
    \item Without reranking
    \item Without causal analysis
    \item Without LLM generation (using template-based responses)
\end{itemize}

\section{Results}

\subsection{Quantitative Results}

[Results will be filled in after running experiments]

\subsubsection{Task 1 Performance}
\begin{itemize}
    \item Average response quality scores
    \item Evidence retrieval precision/recall
    \item Explanation completeness metrics
\end{itemize}

\subsubsection{Task 2 Performance}
\begin{itemize}
    \item Conversational coherence scores
    \item Context usage effectiveness
    \item Follow-up detection accuracy
\end{itemize}

\subsection{Qualitative Analysis}

[Qualitative examples will be included]

\subsubsection{Example Explanations}
We provide examples of:
\begin{itemize}
    \item High-quality explanations with strong evidence
    \item Explanations showing system limitations
    \item Follow-up conversation flows
\end{itemize}

\subsection{Error Analysis}

[Error analysis will be included]

Common error types:
\begin{itemize}
    \item Insufficient evidence retrieval
    \item Misinterpretation of causal relationships
    \item Context loss in follow-up conversations
\end{itemize}

\section{Limitations and Future Work}

\subsection{Limitations}
\begin{itemize}
    \item Dependency on transcript quality and speaker labeling accuracy
    \item Limited to English language
    \item Computational cost of LLM inference
    \item Need for domain-specific fine-tuning for optimal performance
\end{itemize}

\subsection{Future Work}
\begin{itemize}
    \item Custom model training for domain-specific causal patterns
    \item Multi-language support
    \item Real-time processing capabilities
    \item Enhanced interpretability visualizations
    \item Integration with live call monitoring systems
\end{itemize}

\section{Conclusion}

We presented a comprehensive system for causal rationale extraction from conversational data. The system successfully combines retrieval, causal analysis, and LLM-based generation to provide evidence-based explanations for business events. Evaluation demonstrates the effectiveness of our approach, though there is room for improvement in handling edge cases and domain-specific scenarios.

\bibliographystyle{ieeetr}
\begin{thebibliography}{99}
\bibitem{rag} Lewis, P., et al. (2020). Retrieval-Augmented Generation for Knowledge-Intensive NLP Tasks.
\bibitem{causal} Feder, A., et al. (2021). Causal Inference in Natural Language Processing: Estimation, Prediction, Interpretation and Beyond.
\bibitem{dialogue} Jurafsky, D., \& Martin, J. H. (2020). Speech and Language Processing.
\end{thebibliography}

\end{document}

